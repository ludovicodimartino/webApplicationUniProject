\section{Group Members Contribution}

% For each group member, describe in detail the contribution to the project.

\begin{description}
	\item[Di Martino Ludovico] In this part of the project, I drew the mockup of the web application using Figma. Then I set up the \textit{init\_db.sql} (the sql file that populates the database) in order to make it run after the \textit{wacar.sql} (the file used to create all the database tables). To do this I modified the \textit{docker-compose.yml} and I created the \textit{container\_health\_check.sh} script. Furthermore, I implemented the Admin servlet with all its functionalities: insert car, insert circuit, insert car type, insert circuit type, edit car, edit circuit, insert car-circuit suitability and delete car-circuit suitability. To do so, I also coded the useful DAOs, resource objects and JSP pages for the required functionalities. Finally, I contributed in the creation of the javadoc for the code that I've written.
	\item[Galli Filippo] In this project my primary focus was on developing the homepage management and user operations. This involved implementing components such as the UserServlet and the HomeServlet, as well as crafting corresponding JSP pages (such as login.jsp, signup.jsp, userPage.jsp and home.jsp), implementing the UserRegisterDAO, the UserLoginDAO, and the GetUserByEmailDAO. Additionally, I developed the LoginFilter and the HomeFilter. I have also partially contributed in the creation of the JavaDoc, in the writing of the report and in the shaping of the Entity-Relationship schema and the error code infrastructure.
	\item[Leonardi Alessandro] In this part of the project I contributed to the definition and the development of the Favourite resource, the visualization of the list of favourites and the possibility of deliting this favourites. I started by defining "Favourite.java" class, then "ListFavouriteDAO.java" and "DeleteFavouriteDAO.java", then I have continued by implementing the corresponding servlet "ListFavouriteServlet.java" and "DeleteFavouriteServlet.java". In conclusion I have represented the favourites with list-favourites.jsp where I have merged the visualization and the possibility to delete each Favourite. 
	\item[Rigobello Manuel] In this project, I contributed to the creation and development of the Entity-Relationship schema, the relational schema, to the population of the tables and to the triggers. Then, I started to develop the order page, which includes the jsp page (for the second homework I will improve the flow of creating an order by including the client-side part), the ListCarByAvailabilityDAO, the ListCircuitByCarType, the InsertOrderDAO for inserting the new order into the table, the recap page and the OrderServlet that handles the different url. Next, I developed some rest api. In particular, I started with the dispatcher for retrieving a specific order given its id, and then I focused on the protected api to distinguish between user and admin api. Regarding the documentation, I partially contributed in the creation of the JavaDoc, while in the report I described the Data Logic Layer (section \ref{data_logic_layer}) and the Read Order REST api.
	\item[Scapinello Michele] In this project, I contributed to improving the ER schema by raising questions about its structure and connecting it to the functionalities that our application needed to include.
								I developed DAO objects such as ListCarDAO, ListCircuitDAO, and ListOrdersByEmailDAO, along with their corresponding servlets ListCarServlet, ListCircuitServlet, ListOrdersByEmailServlet, and LoadCircuitImageServlet.
	Additionally, I created the associated JSP files, such as list-car.jsp, list-circuit.jsp, and list-orders.jsp. Following the introduction of the REST paradigm, I also developed the ListCarsRR and ListCircuitsRR resources and handled their invocation within the RestDispatcherServlet.
	I partially contributed, along with other team members, to implementing JavaDoc where necessary and addressing any general issues.
	In the report, I contributed to the Presentation Logic Layer section and the Business Logic Layer section.

\end{description}
