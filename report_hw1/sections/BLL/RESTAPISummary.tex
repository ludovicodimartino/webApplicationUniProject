\subsection{REST API Summary}

%describe the REST API. If needed, add a few lines of text here, describing the content of the table.
Part of the URIs are filtered through different filters. U indicates that only users can access to the endpoint,
A that only administrators can request the specified URI to the server.

\begin{longtable}{|p{.375\columnwidth}|p{.1\columnwidth} |p{.35\columnwidth}|p{.1\columnwidth}|} 
\hline
\textbf{URI} & \textbf{Method} & \textbf{Description} & \textbf{Filter} \\\hline
/rest/circuit & GET & Allows to list all the circuits \\\hline
/rest/order & POST & Allows to process an order \\\hline
/rest/list-car & GET & Allows to list all the cars \\\hline
/user/signup/  & POST &  Allows to register a new student in WaCar & \\\hline
/user/login/  & POST &  Allows to login a USER or ADMIN in WaCar with his credentials & \\\hline
/user/logout/  & POST &  Allows to logout a session from WaCar, and the session is cleared & \\\hline
/user/ & GET &  Returns the information about the user & U \\\hline
/home/  & GET &  Returns the information to be added in the homepage according to the user's role & \\\hline
/car\_list/  & GET &  Returns list of all cars in the database & \\\hline
/circuit\_list/  & GET &  Returns list of all circuits in the database & \\\hline
/admin/insertCar/ & POST &   & A\\\hline
/create-order/cars/ & POST & Shows a list of availabe cars, and the user must select one of these options. & U\\\hline
/create-order/circuits/ & POST & Shows the list of possible circuits where the user can race with the selected car & U\\\hline
/create-order/complete-order/ & POST & Show the form in which the user inserts the date and the number of laps & U\\\hline
/create-order/recap/ & POST & Shows the information of the selected items and the total price & U\\\hline
/create-favourite/recap/ & POST & Shows the information of the new favourite item & U\\\hline

\caption{REST API Table}
\label{tab:termGlossary}
\end{longtable}