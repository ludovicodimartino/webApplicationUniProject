\subsection{REST API Details}

In this section we report all the resources available via the REST API. In particular, we implemented via REST API, all the parts of the web application that allows the final user to look at the cars and the circuits and to make an order (the REST resources have been tested via the curl method).

\subsection*{List circuits}

The following endpoint allows to list all the circuits (and returns a JSON). For this first part of the project, to see the JSON output it is necessary to use the command "\textit{curl http://localhost:8081/wacar/rest/circuit}" from the prompt, because currently the "application/json" media type is not supported. In this way, you can still check the first version of the page implemented with a servlet, while in the next part of the project we will complete the REST version.

\begin{itemize}
    \item URL: \texttt{/rest/circuit}
    \item Method: \texttt{GET}
    \item URL parameters: \texttt{None}
    \item Data parameters: \texttt{None}
    \item Success Response: \\
        \textbf{Code}: 200\\
        \textbf{Content}: The JSON corresponding to the list of circuits
        \texttt{INSERT JSON}
    \item Error Response:\\
    \textbf{Code}: 500 (internal code: E5A1)\\
    \textbf{Content}: The message containing the error (server error when retrieving the circuits)\\
    \begin{verbatim}
    {
        "message": {
        "message": ""Unexpected error while processing a resource."",
        "error-code": "E5A1"
    }}
    \end{verbatim}
    \item Error Response:\\
    \textbf{Code}: 500 (internal code: E5A1)\\
    \textbf{Content}: The message containing the error (server error when retrieving the circuits)\\
    \begin{verbatim}
    {
        "message": {
        "message": ""Unexpected error while processing a resource."",
        "error-code": "E5A1"
    }}
    \end{verbatim}
    
\end{itemize}

\subsection*{List cars}

The following endpoint allows to list all the cars (and returns a JSON). As for the list car REST api, to see the JSON output it is necessary to use the command "\textit{curl http://localhost:8081/wacar/rest/cars}"

\begin{itemize}
    \item URL: \texttt{/rest/cars}
    \item Method: \texttt{GET}
    \item URL parameters: \texttt{None}
    \item Data parameters: \texttt{None}
    \item Success Response: \\
    \textbf{Code}: 200\\
    \textbf{Content}: The JSON corresponding to the list of cars
    \texttt{INSERT JSON}
    \item Error Response:\\
    \textbf{Code}: 500 (internal code: E5A1)\\
    \textbf{Content}: The message containing the error (server error when retrieving the cars)\\
    \begin{verbatim}
    {
        "message": {
        "message": ""Unexpected error while processing a resource."",
        "error-code": "E5A1"
    }}
    \end{verbatim}
    \item Error Response:\\
    \textbf{Code}: 500 (internal code: E5A1)\\
    \textbf{Content}: The message containing the error (server error when retrieving the cars)\\
    \begin{verbatim}
    {
        "message": {
        "message": ""Unexpected error while processing a resource."",
        "error-code": "E5A1"
    }}
    \end{verbatim}

\end{itemize}

\subsection*{Read order}

The following endpoint allows to retrieve a specific order given the id (and returns a JSON). To get the JSON of the order, you have to send an authorization token BASIC in the format user-email:password in Base64.

\begin{itemize}
	\item URL: \texttt{/rest/user/order/\{orderId\}}
	\item Method: \texttt{GET}
	\item URL parameters:
		\\\{orderId\} = the identifier number of the order to be retrieved.
	\item Data parameters: \texttt{None}
	\item Success Response: \\
	\textbf{Code}: 200\\
	\textbf{Content}: The JSON corresponding to the order (the tested URI is: rest/user/order/1):
	\\\begin{verbatim}
	{
		"order":{
				"id":1,
				"account":"default@example.com",
				"date":"2024-05-10",
				"carBrand":"Pagani",
				"carModel": "Zonda R",
				"circuit": "Autodromo Enzo e Dino Ferrari",
				"nLaps":3,
				"price":90,
				"createdAt":"2024-04-25 07:50:06.857"
			}
	}
	\end{verbatim}

	\item Error Response:\\
	\textbf{Code}: 400 (internal code: E4A7)\\
	\textbf{Content}: The message containing the error (wrong URI format)\\
	\begin{verbatim}
		{
			"message": {
				"message": "Wrong format for URI /order/{orderId}: no {orderId} specified.",
				"error-code": "E4A7",
				”error-detail”: ”exception content”
		}}
	\end{verbatim}
	\item Error Response:\\
	\textbf{Code}: 401 (internal code: E4B3)\\
	\textbf{Content}: No authorization header sent by the client\\
	\begin{verbatim}
	{
		"message": {
			"message": "No authorization header sent by the client.",
			"error-code": "E4B3"
	}}
	\end{verbatim}	
	\item Error Response:\\
	\textbf{Code}: 401 (internal code: E4B3)\\
	\textbf{Content}: You are not authorized to perform this operation\\
	\begin{verbatim}
		{
			"message": {
				"message": "You are not authorized to perform this operation",
				"error-code": "E4B3"
		}}
	\end{verbatim}	
	\item Error Response:\\
	\textbf{Code}: 500 (internal code: E5A1)\\
	\textbf{Content}: The message containing the error (server error when retrieving the order)\\
	\begin{verbatim}
		{
			"message": {
				"message": "Cannot retrieve the order: unexpected error.",
				"error-code": "E5A1",
				”error-detail”: ”exception content”
		}}
	\end{verbatim}
	
\end{itemize}
